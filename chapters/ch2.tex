% !TeX spellcheck = it_IT
\chapter{Diagramma delle classi}

\section{Utente}

Interfaccia generale per ogni tipo di utente della piattaforma, contiene le credenziali e i metodi per cambiarle.

\section{Cittadino}

Implementa Utente, contiene le funzionalità per gestire le proposte proprie e interagire con quelle di terzi. Inoltre, un cittadino può eliminare il proprio profilo.

\section{Associazione}

Estende Cittadino con la possibilità di esprimere e rimuovere endorsement.

\section{Moderatore}

Implementa Utente, contiene le funzioni per gestire la coda di moderazione ed eliminare contenuti inappropriati.

\section{Amministratore}

Implementa Utente, consente di creare Associazioni e Moderatori, modificare lo stato di una proposta e scaricare il report dei dati di sistema.

\section{Campi}

Interfaccia con campi basilari.

\section{Votabile}

Un elemento generico che tiene traccia dei propri voti.

\section{BozzaProposta}

Implementa campi, aggiunge informazioni su autore e luogo.

\section{Proposta}

Implementa Votabile, aggiunge stato, versioni e proposte di modifica in attesa di essere valutate.

\section{QuesitoSondaggio}

Implementa Votabile, aggiunge il testo.

\section{Sondaggio}

Implementa campi, aggiunge il periodo di validità e i propri quesiti.

\section{Attività}

Contiene il nome dell'azione compiuta e il timestamp di essa.

\section{Registrazione}

Contiene i dati inseriti durante la registrazione e le funzionalità per verificare che il cittadino sia di Trento, inviare la mail di conferma e infine creare il cittadino.

\begin{center}
    % \includegraphics[scale=1]{img/classi/classe.png}
\end{center}

\section{Diagramma delle classi complessivo}

\begin{center}
    \includegraphics[scale=.45]{img/classi/completo.png}
\end{center}
